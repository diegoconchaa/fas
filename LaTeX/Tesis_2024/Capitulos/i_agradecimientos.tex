Al llegar al cierre de esta etapa tan significativa en mi vida académica y personal, quiero expresar mi más sincero agradecimiento a todas aquellas personas que hicieron posible este logro.

En primer lugar, a mi profesor guía y mentor Michael Miranda, por su invaluable apoyo, guía constante y dedicación durante todo el desarrollo de esta tesis. Su compromiso, exigencia y confianza fueron claves para que este trabajo alcanzara la profundidad y calidad esperadas.

A todo el equipo de desarrollo del proyecto FAS, en especial a su director Sebastián Pérez, por brindarme la oportunidad de participar activamente en un entorno profesional enriquecedor, donde pude aplicar y expandir mis conocimientos.

A mi familia más cercana, quienes han sido un pilar fundamental a lo largo de este proceso: a mi madre María Eugenia Araya, por su amor incondicional y constante aliento; a mi padrino Carlos Chávez, por su apoyo y consejos sabios; y a mi tía y madrina Nena Cerezo, por estar siempre presente con palabras de ánimo y cariño.

También quiero agradecer profundamente a todos los profesores de la Facultad de Ingeniería Mecánica que contribuyeron a mi formación integral. Cada uno, con su vocación y compromiso, dejó una huella en mi camino profesional y personal.

Por último, a mis compañeras y compañeros de la casa de estudios que me acompañaron y apoyaron a lo largo de estos años. Su compañía, colaboración y amistad fueron esenciales para sobrellevar los desafíos y disfrutar los logros del camino recorrido.

A todos y todas, gracias de corazón.