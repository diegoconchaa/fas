\chapter{Anexos}
\thispagestyle{fancy}

\appendix
\renewcommand\thesection{\Alph{section}}
\section{Códigos y Scripts de Modelo Experimental}
En esta sección se incluyen fragmentos de los códigos empleados para la adquisición de datos para una placa ESP32 TTGO T-Display.

\subsection{Script de Configuración del Arduino}

\begin{verbatim}
    #include <WiFi.h>
    #include <WebServer.h>
    #include <TFT_eSPI.h>
    #include <Wire.h>
    #include <Adafruit_AMG88xx.h>
    #include <Adafruit_ADXL345_U.h>    // Replaced MPU6050 library with ADXL345
    #include <Adafruit_Sensor.h>
    #include "xbm.h"                   // Sketch tab header for xbm images
    #include <ArduinoJson.h>           // Include ArduinoJson library
    
    // ===================== PIN DEFINITIONS =====================
    
    // LM35 sensor pin
    #define LM35_PIN       33 // GPIO33 (ADC1_CH5)
    
    // Voltage input pin (if still needed for a generic voltage measurement)
    #define VOLTAGE_PIN    32 // GPIO32 (ADC1_CH4)
    
    // FZ0430 voltage sensor pin (adjust as needed)
    #define FZ0430_PIN     34 // Example: GPIO34 (ADC1_CH6)
    
    // ACS712 current sensor pin (adjust as needed)
    #define ACS712_PIN     36 // Example: GPIO36
    
    // PWM output pin (fan control, etc.)
    #define PWM_PIN        25 // GPIO25
    
    // I2C pins (ESP32 default)
    #define SDA_PIN        21
    #define SCL_PIN        22
    
    // Button pins
    #define BUTTON_LEFT_PIN  0   // GPIO0  - Left Button
    #define BUTTON_RIGHT_PIN 35  // GPIO35 - Right Button
    
    // ===================== OBJECTS & GLOBALS =====================
    TFT_eSPI tft = TFT_eSPI();
    WebServer server(80);
    
    Adafruit_AMG88xx amg;       
    Adafruit_ADXL345_Unified adxl = Adafruit_ADXL345_Unified(12345);
    
    int pwmValue = 0;  // PWM value for fan control
    
    const char* ssid = "Network";      // Replace with your Wi-Fi SSID
    const char* password = "Password"; // Replace with your Wi-Fi password
    
    // Cache for ADXL345 at 100Hz
    sensors_event_t cachedAccelEvent;
    unsigned long lastMPUReadTime = 0;
    const unsigned long mpuReadInterval = 10; // Read every 10 ms (100 Hz)
    
    // Update display
    unsigned long lastDisplayUpdateTime = 0;
    const unsigned long displayUpdateInterval = 500;
    
    // Button read timing
    unsigned long lastButtonReadTime = 0;
    const unsigned long buttonReadInterval = 250;
    
    String ipAddress;
    
    // PWM control variables
    int pwmLevel = 0;             
    const int maxPwmLevel = 10;   
    
    // Button press timing variables
    unsigned long buttonLeftPressedTime = 0;
    unsigned long buttonRightPressedTime = 0;
    const unsigned long longPressDuration = 1000; // 1 second for long press
    
    // Last button states
    bool buttonLeftLastState = HIGH;
    bool buttonRightLastState = HIGH;
    
    // ===================== CORS HELPER =====================
    void sendCORSHeaders() {
      server.sendHeader("Access-Control-Allow-Origin", "*");
      server.sendHeader("Access-Control-Allow-Methods", "GET, POST, OPTIONS");
      server.sendHeader("Access-Control-Allow-Headers", "Content-Type");
    }
    
    // ===================== HANDLERS =====================
    
    // Handle LM35 - returns {"value": <number>}
    void handleLM35() {
      int analogValue = analogRead(LM35_PIN);
      float voltage = analogValue * (3.3 / 4095.0);
      float temperatureC = voltage * 100.0;
    
      StaticJsonDocument<200> jsonResponse;
      jsonResponse["value"] = temperatureC;
    
      String response;
      serializeJson(jsonResponse, response);
    
      sendCORSHeaders();
      server.send(200, "application/json", response);
    }
    
    // Handle Voltage (generic) - returns {"value": <number>}
    void handleVoltage() {
      int analogValue = analogRead(VOLTAGE_PIN);
      float voltage = analogValue * (3.3 / 4095.0);
      float inputVoltage = voltage * (5.0 / 3.3);
    
      StaticJsonDocument<200> jsonResponse;
      jsonResponse["value"] = inputVoltage;
    
      String response;
      serializeJson(jsonResponse, response);
    
      sendCORSHeaders();
      server.send(200, "application/json", response);
    }
    
    // Handle ADXL345 - returns {"x": <number>, "y": <number>, "z": <number>}
    void handleADXL345() {
      StaticJsonDocument<200> jsonResponse;
      jsonResponse["x"] = cachedAccelEvent.acceleration.x;
      jsonResponse["y"] = cachedAccelEvent.acceleration.y;
      jsonResponse["z"] = cachedAccelEvent.acceleration.z;
    
      String response;
      serializeJson(jsonResponse, response);
    
      sendCORSHeaders();
      server.send(200, "application/json", response);
    }
    
    // Handle AMG8833 - returns {"grid": [[8 values], [8 values], ...]}
    void handleAMG8833() {
      float pixels[64];
      amg.readPixels(pixels);
    
      StaticJsonDocument<2048> jsonResponse; // Larger document for 8x8 data
      JsonArray grid = jsonResponse.createNestedArray("grid");
    
      for (int row = 0; row < 8; row++) {
        JsonArray rowArray = grid.createNestedArray();
        for (int col = 0; col < 8; col++) {
          rowArray.add(pixels[row * 8 + col]);
        }
      }
    
      String response;
      serializeJson(jsonResponse, response);
    
      sendCORSHeaders();
      server.send(200, "application/json", response);
    }
    
    // Handle setPWM - sets PWM (0-255)
    void handleSetPWM() {
      if (server.hasArg("value")) {
        sendCORSHeaders();
        String valueStr = server.arg("value");
        int value = valueStr.toInt();
        if (value >= 0 && value <= 255) {
          pwmValue = value;
          analogWrite(PWM_PIN, pwmValue);
          server.send(200, "text/plain", "PWM value set to " + String(pwmValue));
        } else {
          server.send(400, "text/plain", "Invalid PWM value. Must be between 
          0 and 255.");
        }
      } else {
        server.send(400, "text/plain", "PWM value not provided.");
      }
    }
    
    // ===================== NEW SENSOR HANDLERS =====================
    
    /**
     * Handle FZ0430 Voltage Sensor
     * Returns {"value": <voltage>}
     * 
     * Adjust the multiplication factor based on your actual FZ0430 ratio.
     */
    void handleFZ0430() {
      int rawValue = analogRead(FZ0430_PIN);
      // Convert raw ADC reading to voltage (3.3V range, 12-bit ADC => max 4095)
      float measuredVoltage = rawValue * (3.3f / 4095.0f);
    
      // FZ0430 modules often step down higher voltages to ~3.3V range.
      // Adjust the scaling factor to match your sensor's specs.
      // For example, if the sensor divides input voltage by 5:
      float actualVoltage = measuredVoltage * 5.0f;
    
      StaticJsonDocument<200> doc;
      doc["value"] = actualVoltage;
    
      String response;
      serializeJson(doc, response);
    
      sendCORSHeaders();
      server.send(200, "application/json", response);
    }
    
    /**
     * Handle ACS712 Current Sensor
     * Returns {"value": <current>}
     * 
     * Adjust offset and sensitivity based on your ACS712 variant:
     *  - ACS712 5A  => 185 mV/A
     *  - ACS712 20A => 100 mV/A
     *  - ACS712 30A => 66 mV/A
     */
    void handleACS712() {
      int rawValue = analogRead(ACS712_PIN);
      // Convert raw ADC reading to voltage
      float sensorVoltage = rawValue * (3.3f / 4095.0f);
    
      // ACS712 outputs Vcc/2 at 0 A. For 3.3V supply, that's ~1.65 V offset.
      float offset = 3.3f / 2.0f;  
      // Sensitivity in V/A (adjust as appropriate)
      float sensitivity = 0.185f; // for a 5A version
    
      float current = (sensorVoltage - offset) / sensitivity;
    
      StaticJsonDocument<200> doc;
      doc["value"] = current;
    
      String response;
      serializeJson(doc, response);
    
      sendCORSHeaders();
      server.send(200, "application/json", response);
    }
    
    // ===================== SETUP =====================
    void setup() {
      Serial.begin(115200);
    
      tft.init();
      tft.setRotation(0);
      tft.fillScreen(TFT_BLACK);
      tft.setTextColor(TFT_WHITE, TFT_BLACK);
      tft.setTextSize(1);
      tft.setTextWrap(true);
    
      Wire.begin(SDA_PIN, SCL_PIN);
      Wire.setClock(100000);
    
      tft.drawXBitmap(0, 240 - logoHeight - 25, logo, logoWidth,
      logoHeight, TFT_BLUE);
    
      pinMode(LM35_PIN, INPUT);
      pinMode(VOLTAGE_PIN, INPUT);
      pinMode(FZ0430_PIN, INPUT);
      pinMode(ACS712_PIN, INPUT);
      pinMode(PWM_PIN, OUTPUT);
    
      // Initialize AMG8833
      if (!amg.begin()) {
        Serial.println("Failed to initialize AMG8833!");
        tft.println("AMG8833 init failed!");
      } else {
        Serial.println("AMG8833 initialized.");
      }
    
      // Initialize ADXL345
      if (!adxl.begin()) {
        Serial.println("Failed to initialize ADXL345!");
        tft.println("ADXL345 init failed!");
      } else {
        Serial.println("ADXL345 initialized.");
      }
    
      pinMode(BUTTON_LEFT_PIN, INPUT_PULLUP);
      pinMode(BUTTON_RIGHT_PIN, INPUT_PULLUP);
    
      // Wi-Fi connection
      WiFi.begin(ssid, password);
      tft.println("Connecting to WiFi...");
      while (WiFi.status() != WL_CONNECTED) {
        delay(2000);
        Serial.print(".");
      }
      tft.println("WiFi connected!");
      ipAddress = WiFi.localIP().toString();
      tft.println(ipAddress);
      Serial.println(ipAddress);
    
      // ===================== ENDPOINTS =====================
      server.on("/lm35", handleLM35);
      server.on("/amg8833", handleAMG8833);
      server.on("/adxl345", handleADXL345);
      server.on("/voltage", handleVoltage);
      server.on("/setPWM", handleSetPWM);
    
      // New sensor endpoints
      server.on("/fz0430", handleFZ0430);
      server.on("/acs712", handleACS712);
    
      // OPTIONS handlers for CORS preflight if needed
      server.on("/lm35", HTTP_OPTIONS, []() {
        sendCORSHeaders();
        server.send(204);
      });
      server.on("/amg8833", HTTP_OPTIONS, []() {
        sendCORSHeaders();
        server.send(204);
      });
      server.on("/adxl345", HTTP_OPTIONS, []() {
        sendCORSHeaders();
        server.send(204);
      });
      server.on("/voltage", HTTP_OPTIONS, []() {
        sendCORSHeaders();
        server.send(204);
      });
      server.on("/setPWM", HTTP_OPTIONS, []() {
        sendCORSHeaders();
        server.send(204);
      });
      server.on("/fz0430", HTTP_OPTIONS, []() {
        sendCORSHeaders();
        server.send(204);
      });
      server.on("/acs712", HTTP_OPTIONS, []() {
        sendCORSHeaders();
        server.send(204);
      });
    
      server.begin();
      Serial.println("HTTP server started");
    }
    
    // ===================== LOOP =====================
    void loop() {
      server.handleClient();
    
      unsigned long currentTime = millis();
    
      // ========== Handle Buttons ==========
      if (currentTime - lastButtonReadTime >= buttonReadInterval) {
        lastButtonReadTime = currentTime;
    
        bool buttonLeftState = digitalRead(BUTTON_LEFT_PIN);
        bool buttonRightState = digitalRead(BUTTON_RIGHT_PIN);
    
        // Handle Left Button
        if (buttonLeftLastState == HIGH && buttonLeftState == LOW) {
          buttonLeftPressedTime = currentTime;
        } else if (buttonLeftLastState == LOW && buttonLeftState == HIGH) {
          if (currentTime - buttonLeftPressedTime >= longPressDuration) {
            // Long press => reset PWM to 0
            pwmLevel = 0;
          } else {
            // Short press => decrease PWM level
            if (pwmLevel > 0) {
              pwmLevel--;
            }
          }
          pwmValue = (pwmLevel * 255) / maxPwmLevel;
          analogWrite(PWM_PIN, pwmValue);
        }
    
        // Handle Right Button
        if (buttonRightLastState == HIGH && buttonRightState == LOW) {
          buttonRightPressedTime = currentTime;
        } else if (buttonRightLastState == LOW && buttonRightState == HIGH) {
          if (currentTime - buttonRightPressedTime >= longPressDuration) {
            // Long press => set PWM to max
            pwmLevel = maxPwmLevel;
          } else {
            // Short press => increase PWM level
            if (pwmLevel < maxPwmLevel) {
              pwmLevel++;
            }
          }
          pwmValue = (pwmLevel * 255) / maxPwmLevel;
          analogWrite(PWM_PIN, pwmValue);
        }
    
        buttonLeftLastState = buttonLeftState;
        buttonRightLastState = buttonRightState;
      }
    
      // ========== Update ADXL345 Data ==========
      if (currentTime - lastMPUReadTime >= mpuReadInterval) {
        lastMPUReadTime = currentTime;
        sensors_event_t accelEvent;
        adxl.getEvent(&accelEvent);
        cachedAccelEvent = accelEvent;
      }
    
      // ========== Update Display ==========
      if (currentTime - lastDisplayUpdateTime >= displayUpdateInterval) {
        lastDisplayUpdateTime = currentTime;
        tft.fillScreen(TFT_BLACK);
        tft.setCursor(0, 0);
    
        // ---- LM35 ----
        int lm35Raw = analogRead(LM35_PIN);
        float lm35Voltage = lm35Raw * (3.3f / 4095.0f);
        float temperatureC = lm35Voltage * 100.0f;
        tft.printf("LM35 Temp: %.2f C\n", temperatureC);
    
        // ---- Original "Voltage" reading (if used) ----
        int analogValue = analogRead(VOLTAGE_PIN);
        float voltage = analogValue * (3.3f / 4095.0f);
        float inputVoltage = voltage * (5.0f / 3.3f);
        tft.printf("Voltage: %.2f V\n", inputVoltage);
    
        // ---- FZ0430 Voltage Sensor ----
        int fzRaw = analogRead(FZ0430_PIN);
        float measuredVoltageFZ = fzRaw * (3.3f / 4095.0f);
        float fzVoltage = measuredVoltageFZ * 5.0f;
        tft.printf("FZ0430: %.2f V\n", fzVoltage);
    
        // ---- ACS712 Current Sensor ----
        int acsRaw = analogRead(ACS712_PIN);
        float sensorVoltageACS = acsRaw * (3.3f / 4095.0f);
        float offset = 3.3f / 2.0f;     // 1.65 V for 3.3 V supply
        float sensitivity = 0.185f;    // Example for ACS712 (5A version)
        float currentACS = (sensorVoltageACS - offset) / sensitivity;
        tft.printf("ACS712: %.2f A\n", currentACS);
    
        // ---- ADXL345 Cached Data ----
        tft.printf("ADXL345 Accel:\nX: %.2f\nY: %.2f\nZ: %.2f\n",
                   cachedAccelEvent.acceleration.x,
                   cachedAccelEvent.acceleration.y,
                   cachedAccelEvent.acceleration.z);
    
        // ---- AMG8833 Average Temp ----
        float pixels[64];
        amg.readPixels(pixels);
        float avgTemp = 0;
        for (int i = 0; i < 64; i++) {
          avgTemp += pixels[i];
        }
        avgTemp /= 64.0f;
        tft.printf("AMG8833: %.2f C\n", avgTemp);
    
        // ---- PWM ----
        float speedPWM = 100.0f * pwmValue / 255.0f;
        tft.printf("PWM: %.2f%%\n\n", speedPWM);
    
        // ---- Logo & IP ----
        tft.drawXBitmap(0, 240 - logoHeight - 25, logo, logoWidth, 
        logoHeight, TFT_BLUE);
        tft.println("\n\n\n\n\n\n\n\n\n\n\n\n\n\n\n\n\nIP: " + ipAddress);
      }
    }    
\end{verbatim}

\subsection{Script de Incialización de Base de Datos SQLite}

\begin{verbatim}
  const sqlite3 = require('sqlite3').verbose();

// Connect to the SQLite database
const db = new sqlite3.Database('./database.sqlite', (err) => {
    if (err) {
        console.error('Error opening database:', err.message);
    } else {
        console.log('Connected to the SQLite database.');
    }
});

// Create the 'data' table
db.serialize(() => {
    db.run(`
        CREATE TABLE IF NOT EXISTS data (
        id INTEGER PRIMARY KEY AUTOINCREMENT,
        timestamp TEXT NOT NULL,
        sensor TEXT NOT NULL,
        "values" TEXT NOT NULL
        );
    `, (err) => {
        if (err) {
            console.error('Error creating table:', err.message);
        } else {
            console.log('Table "data" is ready.');
        }
    });
});

// Close the database connection
db.close((err) => {
    if (err) {
        console.error('Error closing database:', err.message);
    } else {
        console.log('Closed the database connection.');
    }
});

\end{verbatim}


\subsection{Script de Configuración Servidor de Datos}

\begin{verbatim}
  // server.js

const express = require('express');
const sqlite3 = require('sqlite3').verbose();
const bodyParser = require('body-parser');
const cors = require('cors');

const app = express();
const PORT = 3000; // Change as needed

// Middleware
app.use(bodyParser.json());
app.use(cors());

// Connect to SQLite database with optimized settings
const db = new sqlite3.Database('./database.sqlite', (err) => {
    if (err) {
        console.error('Error opening database:', err.message);
        process.exit(1); // Exit if database connection fails
    } else {
        console.log('Connected to the SQLite database.');

        // Set PRAGMA settings for performance
        db.serialize(() => {
            db.run("PRAGMA journal_mode = WAL;");
            db.run("PRAGMA synchronous = NORMAL;");
            db.run("PRAGMA cache_size = 100000;"); // Adjust based 
            on available memory
            db.run("PRAGMA foreign_keys = OFF;");
        });

        // Create the 'data' table with escaped 'values' column
        db.run(`
            CREATE TABLE IF NOT EXISTS data (
                id INTEGER PRIMARY KEY AUTOINCREMENT,
                timestamp TEXT NOT NULL,
                sensor TEXT NOT NULL,
                "values" TEXT NOT NULL
            );
        `, (err) => {
            if (err) {
                console.error('Error creating table:', err.message);
                process.exit(1); // Exit if table creation fails
            } else {
                console.log('Table "data" is ready.');
            }
        });
    }
});

// Prepare the INSERT statement with escaped 'values' column
const insertStmt = db.prepare(`
    INSERT INTO data (timestamp, sensor, "values")
    VALUES (?, ?, ?)
`);

// Endpoint to receive sensor data
app.post('/api/sensor-data', (req, res) => {
    const { timestamp, sensor, values } = req.body;

    // Basic validation
    if (!timestamp || !sensor || !values) {
        return res.status(400).json({ error: 'Missing required fields: 
        timestamp, sensor, values' });
    }

    // Ensure timestamp format 'YYYY-MM-DD HH:MM:SS.SSS'
    const timestampRegex = /^\d{4}-\d{2}-\d{2} \d{2}:\d{2}:\d{2}\.\d{3}$/;
    if (!timestampRegex.test(timestamp)) {
        return res.status(400).json({ error: 'Invalid timestamp format. 
        Expected "YYYY-MM-DD HH:MM:SS.SSS"' });
    }

    // Convert values object to JSON string
    let valuesString;
    try {
        valuesString = JSON.stringify(values);
    } catch (err) {
        return res.status(400).json({ error: 
        'Invalid JSON in values field.' });
    }

    // Insert data into the database using the prepared statement
    insertStmt.run([timestamp, sensor, valuesString], function(err) {
        if (err) {
            console.error('Error inserting data:', err.message);
            return res.status(500).json({ error: 'Failed to 
            store data in the database.' });
        }

        // Success response
        res.status(201).json({ message: 'Data stored successfully.',
         id: this.lastID });
    });
});

// Root endpoint (optional)
app.get('/', (req, res) => {
    res.send('Sensor Data API is running.');
});

// Start the server
app.listen(PORT, () => {
    console.log(`Server is running on port ${PORT}.`);
});

\end{verbatim}
